\section{Processor design and VHDL}
In processor design, thousands of transistors have to be carefully arranged.
Every single connection can make or break the design, thus manual placement and
wiring is obviously no option.  However through the use of Hardware Description
Languages (HDL), designers can specify the semantics of their circuits through a
programming-like syntax, and allow a software tool to synthesize the HDL code
into a low-level schematic of the circuit, greatly simplifying the process and
improving efficiency.


VHDL is one such language.  It was originally developed at the hands of the U.S
Department Of Defence to document the behavior of Integrated Circuits (ICs), but
was later expanded to allow logic simulation and synthesis. The language is
documented in the IEEE Standard 1076, and the last revision was added in
2008.[1]

VHDL is used not only to synthesize hardware design, but as it's a complete
programming language, it is used for simulation of test benches as well. When
writing code that is to be synthesized this must be acounted for, as normal
programming paradigms will make the code unsynthesizable.


\section{FPGA}
Field-Programmable Gate Arrays (FPGAs) are ICs where the logic design can be
reconfigured after manufacturing.  They mainly consists of Look-Up tables
(LUTs), which can be implement any logical expressions, and configurable wiring.
Many also contains Block RAM (BRAM) and hardware accelerators in the form of
adder/subtractors, multipliers and more.  Although FPGAs are not comparable to
Application-Specific ICs (ASICs) in raw performance, they allow for quick and
easy testing and remedation of errors, in addition to being cheaper by far due
to mass-production.


\section{MIPS}
MIPS (acronym for Multiprocessor without Interlocked Pipeline Stage), is a RISC
architecture. The instructions are all coded as 32 bit words, with a small
number of different formats, R, I and J (for register, immediate and jump). 

\begin{figure}[ht]
    \centering
    \begin{tabular}{ | c | c | c | c | c | c | }
        \hline
        op & rs & rt & rd & shamt & func \\
        \hline
    \end{tabular}
    \caption{\label{fig:rInstruction}Instruction format of R-instruction}
\end{figure}

The R-instruction takes three registers, and is used like the following code
snippet:

\begin{verbatim}
    rs = rt + rd
\end{verbatim}

The plus is just for reading convience, and can be exchanged with a different
opcode and/or func value. The shamt field tells how many positions to shift,
when used with a shift function.

\begin{figure}[ht]
    \centering
    \begin{tabular}{ | c | c | c |  c  | }
        \hline
        op & rs & rt & constant or address   \\
        \hline
    \end{tabular}
    \caption{\label{fig:iInstruction}Instruction format of I-instruction}
\end{figure}

The immediate instruction is used for immediate aritmethic and load/store
instructions, as well as branching. 


\begin{figure}[ht]
    \centering
    \begin{tabular}{ | c | c c c c c c c c c | }
        \hline
        op & & & & & address & & & &  \\
        \hline
    \end{tabular}
    \caption{\label{fig:jInstruction}Instruction format of J-instruction}
\end{figure}

The J-instruction is used for unconditional jumps.


Sources:

http://en.wikipedia.org/wiki/VHDL

http://en.wikipedia.org/wiki/Field-programmable_gate_array

http://www.bitweenie.com/listings/verilog-vs-vhdl/ [?]

[1] IEEE Standard 1076-2008
%(http://ieeexplore.ieee.org/stamp/stamp.jsp?tp=&arnumber=4772740)

MIPS-lecture (\#3)
