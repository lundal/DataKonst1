In processor design, thousands of transistors have to be carefully arranged.
Every single connection can make or break the design, thus manual placement and wiring is obviously no option.
However through the use of Hardware Description Languages (HDL), designers can specify the semantics of their circuits through a programming-like syntax,
and allow a software tool to synthesize the HDL code into a low-level schematic of the circuit, greatly simplifying the process and improving efficiency.


VHDL is one such language.
It was originally developed at the hands of the U.S Department Of Defence to document the behavior of Integrated Circuits (ICs), but was later expanded to allow logic simulation and synthesis.
[TODO]


Field-Programmable Gate Arrays (FPGAs) are ICs where the logic design can be reconfigured after manufacturing.
They mainly consists of Look-Up tables (LUTs), which can be implement any logical expressions, and configurable wiring.
Many also contains Block RAM (BRAM) and hardware accelerators in the form of adder/subtractors, multipliers and more.
Although FPGAs are not comparable to Application-Specific ICs (ASICs) in raw performance, they allow for quick and easy testing and remedation of errors, in addition to being cheaper by far due to mass-production.


Sources:

http://en.wikipedia.org/wiki/VHDL

http://en.wikipedia.org/wiki/Field-programmable_gate_array

http://www.bitweenie.com/listings/verilog-vs-vhdl/ [?]
