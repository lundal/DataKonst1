\section{Instruction Set}

While no more instructions than those required were implemented in the processor, it would be a simple matter to implement further instructions by adding more condition-blocks to the case statement of the decoder.

\section{Stall state}

It was suggested to use a stall state for both load and store instructions, but we found that stalls were unnecessary for store instructions, as storing data to the memory only takes one clock cycle, opposed to retrieving data and then writing to registries, which takes two.

\section{Problems}

\subsection{Program Counter}

The program counter has to be updated before the Fetch state in the suggested design. We tried to update it during the Execute state, but this caused some trouble for load instructions, as the destination register would be changed to that of the next instruction. We therefore decided to add a flip-flop after the instruction memory to control its output. However, this also meant that we had to move the register for the program counter, otherwise it would require two clock cycles to fetch the instruction.

\subsection{Loading the design on to the FPGA}

As we were about to test our design on the FPGA, some problems arised. We were
unable to generate a functional .bit file, as whenever we tried to upload it to
the FPGA, 
